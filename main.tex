\documentclass{beamer}
\usepackage{graphicx}
\usepackage{enumitem}  
\graphicspath{ {./img/} }
\usetheme{Warsaw}
\usepackage{polski}
\usepackage[utf8]{inputenc}

\title{Android Studio}
\definecolor{green}{rgb}{.125,.5,.25}
\usecolortheme[named=green]{structure}
\subtitle{Po co? Dlaczego? Co innego?}
\author{Sielańczyk Jakub}
\date{} 

\begin{document}

    \begin{frame}
        \titlepage
        \begin{center} \vspace{-50pt}
            \includegraphics[scale=0.2]{aslogo}
         \end{center}
    \end{frame}

    \begin{frame}
        \title{Historia}
        \author{}
        \date{}
        \subtitle{}
        \titlepage
    \end{frame}

    \begin{frame}[t]{Historia} \vspace{10pt}
        \begin{description}
            \item[$\bullet$ Prezentacja 0.1v]- 16.05.2013 - Konwerencja Google
            \item[$\bullet$ Pierwsza stabilna wersja] - 01.12.2014
            \item[$\bullet$ Wersja Javy] - ?
            \item[$\bullet$ Wspierane API] - 21
          \end{description}
          \begin{center}
             \includegraphics[scale=0.15]{androidStudioWebsite}
          \end{center}
    \end{frame}

    \begin{frame}
        \title{Skład Android Studio}
        \author{}
        \date{}
        \subtitle{}
        \titlepage
    \end{frame}

    \begin{frame}[t]{Skład Android Studio (skrócony)}
          \begin{description}
            \item[$\bullet$ SDK]- Android Software Development Kit
            \item[$\bullet$ NDK]- Android Native Development Kit
            \item[$\bullet$ ADB]- Android Debug Bridge  
            \item[$\bullet$ AVD]- Android Virtual Device
            \item[$\bullet$ Layout Editor]- Simple XML editor
            \item[$\bullet$ Gradle]- Build system   
            \item[$\bullet$ IntelliJ IDEA]- Baza Android Studio
            \item[$\bullet$ Profiler]- Measure app performance
          \end{description}
          \begin{center}
            \includegraphics[scale=0.2]{aslogo}
          \end{center}
    \end{frame}

    \begin{frame}[t]{Android SDK} 
        \begin{block}{Czym jest SDK?}
            Zestaw narzędzi dla programistów przeznaczony do tworzenia aplikacji na platformę Android. 
            Składa się z dwóch części: 
            \newline SDK Tools – Część wymagana do tworzenia aplikacji niezależnie od wersji Androida
            \newline Platform Tools – Narzędzia zmodyfikowane pod kątem konkretnej wersji systemu
        \end{block}

        \begin{center}\vspace{0pt}
            \includegraphics[scale=0.3]{sdkLogo}
          \end{center}

    \end{frame}

    \begin{frame}[t]{Android NDK} 
        \begin{block}{Czym jest NDK?}
            Android NDK to zestaw narzędzi, który umożliwia implementowanie części aplikacji 
            w kodzie natywnym przy użyciu języków takich jak C i C ++ 
  \end{block}

        \begin{center}\vspace{0pt}
            \includegraphics[scale=0.3]{Android-NDK}
          \end{center}

    \end{frame}

    \begin{frame}[t] \vspace{0pt}
        \begin{block}{def}
            dsadsa
        dsadas
        \end{block}
    \end{frame}

\end{document}